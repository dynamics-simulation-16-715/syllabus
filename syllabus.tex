\documentclass[11pt,letterpaper]{article}

\usepackage[margin=1in]{geometry}
\usepackage{termcal}
\usepackage{enumitem}
\usepackage[colorlinks=true, allcolors=blue]{hyperref}
\usepackage{color}
\usepackage{multirow}
\usepackage{multicol}

\newcommand{\todo}[1]{\textcolor{red}{TODO: #1}}

\title{16-715: Advanced Robot Dynamics and Simulation}
\author{Fall 2022}
\date{}

\begin{document}

\maketitle

\section*{Course Description}

This course explores the fundamental mathematics behind modeling the physics of robots, as well as state-of-the-art algorithms for robot simulation. We will review classical topics like Lagrangian mechanics and Hamilton's Principle of Least Action, as well as modern computational methods like discrete mechanics and fast linear-time algorithms for dynamics simulation. A particular focus of the course will be rigorous treatments of 3D rotations and non-smooth contact interactions (impacts and friction) that are so prevalent in robotics applications. We will use numerous case studies to explore these topics, including quadrotors, fixed-wing aircraft, wheeled vehicles, quadrupeds, humanoids, and manipulators. Homework assignments will focus on practical implementation of algorithms and a course project will encourage students to apply simulation methods to their own research.

\medskip
\noindent
\textbf{Prerequisites:} Strong linear algebra skills, experience with a high-level programming language like Python, MATLAB, or Julia, and basic familiarity with ordinary differential equations.

\section*{Instructors}

\begin{center}
\begin{tabular}{l l}
	Prof. Zac Manchester & \textbf{Email:} \href{mailto:zacm@cmu.edu}{zacm@cmu.edu} \\
	TA: Anthony Wertz & \textbf{Email:} \href{mailto:awertz@cmu.edu}{awertz@cmu.edu}
\end{tabular}
\end{center}

\section*{Logistics}

\begin{itemize}
	\item Lectures will be held Tuesdays and Thursdays 10:10--11:30 AM Eastern time in NSH 1305.
	\item Office hours will be Wednesdays 11:30-12:30 PM Eastern time in NSH 1505, and Fridays 9-10 AM Eastern time on zoom. Zoom link and schedule:   \url{https://tinyurl.com/98k5hs8w}.
	\item Homework assignments will be due by 11:59 PM Eastern time on Wednesdays. Two weeks will be given to complete each assignment.
	\item GitHub will be used to distribute and collect assignments.
	\item Slack will be used for general discussion and Q\&A outside of class and office hours.
	\item There will be no exams. Instead, each student will complete a project on a topic of their choice.
\end{itemize}

\section*{Learning Objectives}
By the end of this course, students should be able to:
\begin{enumerate}
	\item Derive differential equations for simple mechanical systems using classical Newton-Euler and Lagrangian techniques
	\item Model more complex industrial, wheeled, legged, aerial, underwater, and space robotic systems using modern computational techniques
	\item Simulate environmental contact interactions like impacts and Coulomb friction
	\item Understand the trade-offs and limitations of different dynamics formulations and simulation techniques
	\item Build high-performance simulation tools that can be deployed in machine learning and control design applications
	
\end{enumerate}

\section*{Learning Resources}

There is no textbook required for this course. Video recordings of lectures and lecture notes will be posted online. Additional references for further reading will be provided with each lecture.

\section*{Homework}

Homework will be posted every 2 weeks and students will be given at least one full week to complete assignments. All homework will be distributed and collected using GitHub. Solutions and grades will be returned within one week of homework due dates.

\section*{Grading}

Grading will be based on:
\begin{itemize}
	\item 50\% Project
	\item 40\% Homeworks
	\item 10\% Participation
\end{itemize}
Attendance during lectures is not required to earn a full participation grade. Students can also participate through any combination of office hours, Slack discussions, project presentations, and by offering constructive feedback about the course to the instructors.


\section*{Course Policies}

\textbf{Late Homework:} Students are allowed a budget of 6 late days for turning in homework with no penalty throughout the semester. They may be used together on one assignment, or separately on two assignments. Beyond these six days, no other late homework will be accepted.

\medskip
\noindent
\textbf{Accommodations for Students with Disabilities:} If you have a disability and are registered with the Office of Disability Resources, I encourage you to use their online system to notify me of your accommodations and discuss your needs with me as early in the semester as possible. I will work with you to ensure that accommodations are provided as appropriate. If you suspect that you may have a disability and would benefit from accommodations but are not yet registered with the Office of Disability Resources, I encourage you to contact them at \href{mailto:access@andrew.cmu.edu}{access@andrew.cmu.edu}.

\medskip
\noindent
\textbf{Statement of Support for Students' Health \& Well-Being:} Take care of yourself. Do your best to maintain a healthy lifestyle this semester by eating well, exercising, avoiding drugs and alcohol, getting enough sleep, and taking some time to relax. This will help you achieve your goals and cope with stress.

\medskip
\noindent
If you or anyone you know experiences any academic stress, difficult life events, or feelings like anxiety or depression, we strongly encourage you to seek support. Counseling and Psychological Services (CaPS) is here to help: call 412-268-2922 and visit \href{http://www.cmu.edu/counseling}{http://www.cmu.edu/counseling}. Consider reaching out to a friend, faculty, or family member you trust for help getting connected to the support that can help.

\medskip
\noindent
\textit{If you or someone you know is feeling suicidal or in danger of self-harm, call someone immediately, day or night:}

\textit{CaPS: 412-268-2922}

\textit{Re:solve Crisis Network: 888-796-8226}

\medskip
\noindent
\textit{If the situation is life threatening, call the police:}

\textit{On campus: CMU Police: 412-268-2323}

\textit{Off campus: 911}

\medskip
\noindent
If you have questions about this or your coursework, please let me know. Thank you, and have a great semester.


\section*{Tentative Schedule}

\begin{tabular}{c|c|c|c}
	Week & Dates & Topics & Assignments \\
	\hline
	\multirow{2}{*}{1} & Aug 30 & Course Overview \& Review of Newtonian Mechanics Concepts \\
	 & Sep 1 & Particles, Pendulums, and Orbits &  \\
	\hline
	\multirow{2}{*}{2} & Sep 6 & Energy and Stability &  \\
	 & Sep 8 & Numerical Solution of ODEs and Runge-Kutta Methods &  \\
	\hline
	\multirow{2}{*}{3} & Sep 13 & Runge-Kutta Methods Pt. 2 & HW 1 Out \\
	 & Sep 15 & Rigid Bodies, Euler's Equation, and Lie Groups Pt. 1 &  \\
	\hline
	\multirow{2}{*}{4} & Sep 20 & Rigid Bodies, Euler's Equation, and Lie Groups Pt. 2 &  \\
	 & Sep 22 & Quaternions and Numerical Simulation & \\
	\hline
	\multirow{2}{*}{5} & Sep 27 & Quadrotors, Airplanes, and Spacecraft & HW 1 Due \\
	 & Sep 29 & Constrained Optimization Pt. 1 & HW2 Out \\
	\hline
	\multirow{2}{*}{6}  & Oct 4 & The Least-Action Principle &  \\
	 & Oct 6 & Coordinates, Constraints, and Manifolds & \\
	\hline
	\multirow{2}{*}{7}  & Oct 11 & D'Alembert, Virtual Work, and Generalized Forces &  HW2 Due\\
	 & Oct 13 & Lagrangian Mechanics and Manipulators &  \\
	\hline
	\multirow{2}{*}{8}  & Oct 18 & \textcolor{red}{No Class} & \\
	 & Oct 20 & \textcolor{red}{No Class} & \\
	\hline
	\multirow{2}{*}{8}  & Oct 18 & Lie Algebras and Euler's Equation from a Lagrangian & HW3 Out \\
	 & Oct 20 & Floating-Base Dynamics & \\
	\hline
	\multirow{2}{*}{9}  & Oct 25 & Constrained Optimization Pt. 2 & \\
	 & Oct 27 & Discrete Mechanics and Variational Integrators &   \\
	\hline
	\multirow{2}{*}{10}  & Nov 1 & Momentum, Duality, and the Hamiltonian &  HW3 Due \\
	 & Nov 3 & Constrained Optimization Pt. 3 & HW4 Out\\
	 \hline
	\multirow{2}{*}{11}  & Nov 8 & Impacts as Inequality Constraints & \\
	 & Nov 10 & Coulomb Friction and The Maximum Dissipation Principle  &   \\
	 \hline
	\multirow{2}{*}{12}  & Nov 15 & Dynamics with Contact Pt. 1 & HW4 Due \\
	 & Nov 17 & Dynamics with Contact Pt. 2 &   \\
	 \hline
	\multirow{2}{*}{13}  & Nov 22 & Maximal vs. Minimal Coordinates for Simulation & \\
	 & Nov 24 & \textcolor{red}{No Class} & \\
	 \hline
	 \multirow{2}{*}{14}  & Nov 29 & Fast Simulation Algorithms Pt. 1 &  \\
	 & Dec 1 & Fast Simulation Algorithms Pt. 2 &   \\
	 \hline
	\multirow{2}{*}{14}  & Dec 6 & Project Presentations &  \\
	 & Dec 8 & Project Presentations &   \\
\end{tabular}

\section*{Project Guidelines}

Students should work in groups of 1--4 to complete a substantial final project. The goal is for students to apply the coarse content to their own research. Project proposals will be solicited in late September and topics will be selected in consultation with the instructors.

\medskip
\noindent
Project grades will be based on a short (four-minute ``lightning talk'') presentation given during the last week of class and a final report submitted via \href{https://forms.gle/6uj9E8XkzP8mAh6v9}{Google drive} by December 15 at midnight. Reports should be written in the form of a 6 page (plus references) ICRA or IROS conference paper using the standard \href{https://www.ieee.org/conferences/publishing/templates.html}{two-column IEEE format}. Sections should include an abstract, introduction and/or background to motivate your problem, 2--3 main technical sections on your contributions, conclusions, and references. Grading will be based on the following criteria:
\newline
\newline
\begin{tabular}{|c|l|}
\hline
10\% & Class presentation on December 2nd\\
\hline
10\% & Adherence to IEEE formatting and length requirements \\
\hline
10\% & Innovation \& Creativity: Is what you did new/cool/interesting? Convince me. \\
\hline
30\% & Clarity of presentation: Can I understand what you did from your writing + plots? \\
\hline
40\% & Technical correctness: Are your results reasonable? Is your code correct? \\
\hline	
\end{tabular}


\end{document}
